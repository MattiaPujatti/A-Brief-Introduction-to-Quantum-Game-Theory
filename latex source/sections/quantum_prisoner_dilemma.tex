\hfill

\section{Quantum Prisoners' Dilemma}
In this section will be analyzed a canonical non-zero sum game usually known as the prisoners' dilemma. In these kind of games, in contrast to zero-sum games, the two players no longer appear in strict opposition to each other, but may rather benefit from mutual cooperation. \\
There are two players, A and B, that are thieves caught by the police, which are being questioned and can choose among two strategies, Mum(M) or Fink(F). The payoffs are distributed according table \ref{eq:pris_dilemma} and, as usually, the objective of each player is to maximize his or her individual utility.
\begin{equation}
\begin{blockarray}{cccc}
& & \text{Player B}&\\
& & M & F\\
\begin{block}{cc[cc]}
\text{Player A} & M & 3,3 &  0,5 \\
 & F & 5,0 & 1,1 \\
\end{block}
\end{blockarray}
\label{eq:pris_dilemma}
\end{equation}
Analyzing the problem using classical game theory, one can conclude that the best strategy to be played is found at the Nash Equilibrium, i.e. playing $\{FF\}$. The "dilemma" in the name of the game stands exactly for this: $\{FF\}$ is a strategy that is Pareto dominated by $\{MM\}$, so both players have, at least from a theoretical point of view, the possibility of obtaining an higher payoff simply unilaterally changing their own strategies, but the common rationality instead force them to play $F$.\\
But when the problem is instead reformulated in the quantum context \cite{Eisert_1999}, it can be proven that :
\begin{itemize}[noitemsep]
	\item[-] there exists
a particular pair of quantum strategies which always gives high reward and is a Nash equilibrium;
	\item[-] there exists a particular quantum strategy which always gives a positive reward if played
against any classical strategy.
\end{itemize}
According to the formalism presented in the previous section, assuming that each player is able to use quantum strategies, than the available actions at the beginning of the game are represented by operators in $SU(2)$ (unitary and trace-preserving 2x2 matrices). Moreover, in order to properly construct a quantum game, also a dose of entanglement, adjusted by the parameter $\gamma$, is needed.\\
For $\gamma=0$, according to figure \ref{}, that shows A's expected payoff in this configuration, for any possible choice $\hat{U}_B$ of B, the first player's utility is maximized if A selects to play $\hat{F}$. But since the game in this case is symmetric, the same reasoning can be applied to player B. This means that $\{\hat{F},\hat{F}\}\equiv\hat{F}\otimes\hat{F}$ is an equilibrium in dominant strategies. Indeed separable games ($\gamma=0$), manifest the same behavior as their classical counterpart, bringing also to the same results.\\
The situation is entirely different for a maximally entangled game, with $\gamma=\pi/2$: this time, any pair of strategies chosen will have no counterpart in the classical domain, providing new and interesting solutions. To be precise, one can always re-derive the original prisoners' dilemma if both players choose actions of the type $\hat{U}(\pi/2,0)$.






