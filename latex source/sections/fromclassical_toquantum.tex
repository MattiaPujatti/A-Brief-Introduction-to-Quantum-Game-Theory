\hfill

\section{From Classical to Quantum Game Theory}
Game theory is the study of mathematical models of strategic interaction among rational decision-makers. Such decision-makers are players that have to follow fixed rules and have interests in all the possible outcomes, which do not depend only on the choices they individually make but also from the decisions took by other contenders. At the very beginning, this discipline was mainly focused on the contest of economic theory, until its first formalization provided by Von Neumann and Morgenstern. Subsequently much work was done in this field, transforming it in a mature discipline, that is now used in several areas such as social sciences, political sciences, biology and engineering.\\
However, in the middle of 90's, physicists started applying the rules of quantum mechanics to classical information theory, giving way to birth of a new discipline that nowadays is known as quantum information theory. Some of them, moreover, started thinking also about a possible recast of classical game theory using quantum probability amplitudes, and hence studying the effect of quantum superposition, interference and entanglement on the agents’ optimal strategies.\\
There are several reasons for which quantizing games may be interesting:
\begin{itemize}[noitemsep]
	\item[-] classical game theory is a well defined discipline in applied mathematics with applications in economy, psychology, biology,... and is based on probability to a large extend, and so there is a fundamental interest in generalizing it to the domain of quantum probabilities;
	\item[-] a lot of games require a "nature choice" or even a "nature player", but under this new perspective, such nature is ruled by quantum mechanics;
	\item[-] if the "selfish gene" theory proposed by Dawkins to justify the selfishness behavior of individuals, even when they live in groups, is reality, than someone can speculate about it bringing adversarial games back to a molecular level in which quantum mechanics dictate the rules \cite{ozhigov2013quantum};
	\item[-] there is an intimate connection between game theory and quantum communication: whenever a player passes his decision to another one, he is basically transferring information, and so in a quantum world it is legitimate to start thinking about quantum information transferred. 
\end{itemize} 
Moreover, it will be shown that in most of the cases, a quantum description of a system provides
advantages over the classical situation.\\
The paper starts analyzing zero-sum games, considering just two adversarial players, and will proceed reformulating several "canonical" problems under this new perspective, showing also that the extension of game theory to the quantum world will be quite "natural" and not even so complex. 
