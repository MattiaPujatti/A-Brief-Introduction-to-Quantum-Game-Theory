\hfill

\section{Conclusion}
In this report a brief overview over the formulation of quantum games has been provided, together with the simulation in Python of two canonical problems like the Prisoners' Dilemma and the Battle of Sexes. Is has been shown that quantum games can provide new and unexpected Nash equilibria, can make some classical results to disappear and can improve a lot the degrees of freedom of the players. Moreover, it has been shown that a quantum player, most of the times, has an advantage over the classical player, with new and surprising "always-win" strategies.\\	
An interesting observation about this brach of game theory is that most of the articles, papers, projects, that you can find on the internet have been written around 1999 and 2001, that are also the years in which the discipline of quantum information has been, for the first time, formalized arising the curiosity of many scientists also belonging to different research areas. The fact is that such wave of first interest vanished as it was born, and no other "revolutional" discoveries or studies, have been realized in the following 20 years. As repeated several times, in fact, the formulation of a quantum game is not so challenging once you have some knowledge about group theory and quantum mechanics, but the successive analysis in order to find solutions, equilibria is very complex in a continuous space. One possible frontier is the numerical simulation, as for the problems analyzed here, but using just a classical computer the amount of resources required is enormous. And this is the reason for which apparently in 2020 the interest of scientific community seems to start rising again: the new innovations (especially engineering) provided by big companies in the development of quantum computers allowed scientists to exploit the power of physical qubits to perform simulations with a decent level of accuracy for the results. And here the situation changes completely, making the analysis much faster and precise. Actually, however, there is no quantum computer that has reached the "perfection" yet, and the main problem is always the noise produced by atoms' fluctuation (one can think to something similar to bits corruption), as shown also in a recent paper \cite{kairon2020noisy}, in which the 3-player Dilemma is simulated over a quantum computer, and the results are presented as function of the channels' noise.\\
So to conclude, there are still a lot of directions to expand quantum game theory that have been unexplored yet, mainly because of the lack of practical implementation, but the recent improvements in quantum technology, especially in term of accessibility, will probably soon open a large number of new possibilities that will bring back this forgotten branch of game theory under the sights of the scientific community.